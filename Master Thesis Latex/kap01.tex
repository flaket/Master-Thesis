%%=========================================
\section[Introduksjon]{Introduksjon}
%%=========================================
\subsection*{Bakgrunn \& Motivasjon}

%%=========================================
\subsection*{Problemformulering \& mål}
asd

%%=========================================
\subsection*{Bidrag}
Basert på dette arbeidet ønsker jeg å jobbe videre med en masteroppgave som omhandler brukergrensesnitt i forbindelse med smarte hjem. Spesielt vil det være spennende å utforske alternativer til touch-skjermer og laptop-er for å kommunisere med hjemmet. Basert på resultatene fra brukerstudiene og Rogers argumenter presentert i kapittel seks er jeg overbevist om at vi ikke ønsker et proaktivt hjem hvor brukerene lener seg tilbake og lar omgivelsene ta seg av alt. Jeg tror vi ønsker å bli engasjert og at spennende og utfordrende brukergrensesnitt kan være med på å skape en ny renessanse for interaksjonen med datamaskinen.

Våren 2015 vil jeg jobbe med bruken av gesturer for å interagere med et smart hjem. Dette kan bli i en multimodal setting der man også gjør bruk av touch-skjermer og/eller tale. Oppgaven peker mot at mange mennesker ikke ønsker kameraer i hjemmene sine. Da passer i steden for de infrarøde geste-sensorer presentert i kapittel seks godt. Arbeidet kan omhandle faktiske brukere og utforske om gesturer er en verdifull måte å interagere på. En annen tilnærming kan være å bruke maskinlæring for å få systemet til å lære å tolke ulike gesturer gjennom klassifisering. Et tredje alternativ er å kombinere gesturer med annen input og se om multimodal interaksjon kan forbedre systemets nøyaktighet og avklare misforståelser.
Hovedmålene for dette prosjektet er å
\begin{enumerate}
\item {\color{red} Målbare mål!} Utforske bruken av enkle gestesensorer kombinert med talegjenkjenning som en effektiv og naturlig interaksjon med hjemmet.
\item Utforske kontekstdrevne brukergrensesnitt.
\end{enumerate}

%%=========================================
\subsection*{Disposisjon}
Dette introduksjonskapitellet etterfølges av fire interaksjonsprosjekter, samt et avsluttende kapittel. 