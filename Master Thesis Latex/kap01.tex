%%=========================================
\section[Introduksjon]{Introduksjon}
%%=========================================
\subsection*{Bakgrunn \& motivasjon}
En vårdag noen år fram i tid våkner du av alarmen som går av. Sola skinner inn gjennom soveromsvinduet, der persiennene ble hevet automatisk for tredve minutter siden for å gi en mer naturlig oppvåkning. Du tusler ut på badet, der lyset kommer på idet du entrer rommet. Varmen i baderomsgulvet har vært skrudd av gjennom natten, men har stått på i en tilstrekkelig periode før alarmen gikk av for å tilby en behagelig temperatur. Du vrenger av deg nattskjorta og legger den i skittentøyskurven for hvitt tøy. En sensor registrer at denne kurven nå er blitt full og en hvit klesvask blir automatisk igangsatt. Mens du står og steller deg får du lyst på hjemmebakte rundstykker og kaffe til frokost. Ved å ytre noen ord og utføre noen gester mot en liten plate på baderomsveggen får du skrudd på både ovnen og kaffemaskina på kjøkkenet. Vel ute på kjøkkenet bruker du en touch-skjerm for å finne fram en oppskrift på rundstykker. Underveis i matlagingen blir det nødvendig å lese videre i oppskriften fra touch-skjermen, men hendene dine er nå tilgrisede med deig. Heldigvis håndterer også denne touch-skjermen enkle gester og du blar deg videre i oppskriften ved å sveipe hånda over enheten. Hjemmet holder en kontinuerlig oversikt over matbeholdningen og hva som må handles inn, basert på ønsker om hvilke varer du alltid vil ha tilgjengelig i hjemmet. Etter frokost forlater du hjemmet, på vei til jobb. Døra låses automatisk bak deg og støvsugerroboten setter i gang med å rengjøre huset mens du er borte. Du får så en notifikasjon på telefonen. En app til hjemmet viser en grafisk framstilling av hjemmet og du ser at ovnen fremdeles står på. Du skrur av ovnen via app-en og fortsetter reisen til jobben.

Dette kan bli et vanlig scenario i mange hjem. Teknologien utvikles videre og det er mange områder av hjemmet som kan forenkles og forbedres; automasjon, energibesparing, hjelp til eldre og funksjonshemmede og forbedret interaksjon.\\\\
Jeg vil nå definere noen ord og uttrykk for å fjerne ambiguitet og legge et solid grunnlag for den videre diskusjonen.\\\\
\emph{Smarte hjem} brukes i denne oppgaven som et samlebegrep for boliger der man kan programmere og styre miljøet og apparatene. Smarte hjem tilbyr funksjonalitet som fjernstyring, energieffektivisering, økt komfort, økt sikkerhet, automasjon og generelt enklere bruk. Jeg har valgt å bruke "smarte hjem", i stedet for "smarte hus" for å påpeke at diskusjonen omtaler faste oppholdssteder generelt.\\\\
\emph{Interaksjon} er kommunikasjonen mellom mennesket og datamaskinen, som i denne sammenhengen styrer hjemmet. Interaksjon omfatter både instruksene brukeren gir, men også tilbakemeldingen datamaskinen svarer med.\\\\
\emph{Naturlig} interaksjon føles instinktiv, forståelig, er enkel i bruk og gir rimelige resultater.\\\\
\emph{Effektiv} interaksjon er praktisk i utførelse og gir raske resultater.\\\\
\emph{Intelligente brukergrensesnitt}. Brukergrensesnitt er programvare som gjør det mulig for en bruker å kommunisere med en datamaskin. De fleste kjenner dette som grafiske grensesnitt som tillater direkte manipulasjon gjennom mus, tastatur eller touch. Intelligente brukergrensesnitt (IUI) er en nyere generasjon av grensesnitt, med evner som å kunne tilpasse seg ulike brukere dynamisk, forstå kontekstinformasjon rundt interaksjonen og å tilby hel eller delvis hjelp til å oppnå brukerens mål. Noen intelligente brukergrensesnitt kan også håndtere flere inputkanaler samtidig, såkalt \emph{multimodal} input. Idéen er at multimodal input kan fremme en mer naturlig og effektiv form for interaksjon, for eksempel ved å håndtere at en bruker benytter touch og tale samtidig.

%%=========================================
\subsection*{Problemformulering}
Nå som begrepene er definert kan jeg presentere oppgavens problemformulering.
\newline\newline
\emph{Kan intelligente brukergrensesnitt benyttes for å tilby en mer naturlig og effektiv interaksjon i smarte hjem?}

%%=========================================
\subsection*{Mål}
Hovedmålene for denne oppgaven er å:
\begin{itemize}
\item Benytte maskinlæring for å utvide egenskapene og bruksområdene til enkle sensorer.
\item Utforske nytteverdien og mulighetene til å bruke gester og tale i smarte hjem. 
\item Utfordre nåværende grafiske brukergrensesnitt for smarte hjem, ved å lage et brukergrensesnitt drevet av kontekstinformasjon, framfor interaksjon.
\end{itemize}

%%=========================================
\subsection*{Disposisjon}
Kapittel to vil greie ut om bakgrunnsteori og relatert arbeid i fagområdene oppgaven har tilknytning til. I kapittel tre presenteres et utvalg hypoteser rundt brukergrensesnitt i smarte hjem, hvordan disse kan utforskes med eksperimenter og hvordan disse eksperimentene ble implementert. Kapittel fire beskriver utførelsen og resultatene fra disse eksperimentene. I kapittel fem oppsummeres oppgaven med diskusjon, forslag til videre arbeid og konklusjon.

%%=========================================
\subsection*{Bemerkning}
Det finnes norske ord og uttrykk for flere av domene denne oppgaven svinger innom, men i mange tilfeller er det vanligere, både i akademia og i industrien, å benytte de engelske uttrykkene. Jeg vil derfor unngå de norske uttrykkene der jeg anser det som vanlig å bruke de engelske. En liste med akronymer finnes i appendix \ref{akronymer}.