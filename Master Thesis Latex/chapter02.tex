%%=========================================
\section{Gestegjenkjennelse med fotodioder}
%%=========================================
\subsection{Introduksjon / Problemet (1-2 sider)}
{\color{red} Omskriv: FORKLAR SOM OM VED WHITEBOARDET!}

La oss vurdere problemet å styre det smarte hjemmet på en naturlig og effektiv måte {\color{red} bruk et eksempel}. Brukere ønsker å kunne kontrollere de ulike delene av hjemmet; lys skal skrus av og på, persienner og garasjedører skal opp og ned, temperatur skal stilles, dører skal låses opp og igjen. Den gamle måten å gjøre det på er individuelle knapper og brytere for hver funksjonalitet. Ettersom antallet enheter som kan styres stadig øker ender vi opp med en økende mengde brytere og det oppstår problemer rundt plass, estetikk og vanskeligheter med å lære hvilke brytere som hører til hvilket apparat. En løsning til dette er å samle styringskontrollen i en app, tilgjengelig på nettbrett og smarttelefoner. {\color{red} naturlig? effektivt?}. Bruken av gester for styring er utforsket i en rekke prosjekter. Tilnærmingen blant disse er RGB-kameraer, dybdekameraer og algoritmer for datasyn. Sammen kan disse prosessere bildene og danne grunnlaget for et system som blant annet kan gjenkjenne avanserte gester. Vel, bruken av kameraer har visse problemer i hjemmescenariet. Brukerundersøkelser {\color{red} refererfef} har vist at de færreste ønsker overvåking i smarte hjem. Selv dersom det kan garanteres at dataene fra kameraene holdes lokalt kan følelsen av at personvernet er utsatt være nok til at brukerene vil holde seg unna.

Dette kapittelet har følgende bidrag:
\begin{itemize}
\item Å presentere enkle fotodioder som et tilstrekkelig medium for enkel brukerinteraksjon {\color{red} fwd ref}.
\item Å vise at fire fotodioder pakket som en gestesensor gjennom maskinlæring kan forstå minimum 10 ulike gester med 95 prosent suksess {\color{red} fwd ref}.
\end{itemize}

\subsection{Min idé (2 sider)}
Min idé har vært å undersøke hvorvidt man kan hente data fra en svært enkel sensor og ved hjelp av standard klassifiseringsalgoritme lage et system som forstår en rekke enkle, men forskjellige gester. Dersom dette kan oppnås kun med dataene samlet fra en sensor på størrelse med en halv fingernegl kan en enkelt sensor ta svært liten plass på veggen i et smart hjem, men samtidig være kraftig nok til å forstå de fleste styringskommandoer som trengs. 

\subsection{Detaljer (5 side)}
Idéen fungerer! Detaljer og data.

\subsubsection{APDS9960 - Gestursensor}
Sensorinfo, sparkfun, arduino

\subsubsection{Overvåket læring - Klassifisering}
asd

\subsubsection{Data-prosessering}
Hente data fra sensoren. Hva slags type data? Hva må gjøres av preprossesering? Valg av algoritmer.

\subsubsection{Støttevektormaskiner}
asd

\subsubsection{K-nærmeste-naboer}
asd

\subsubsection{Ensemble-metoder}
asd

\subsubsection{Resultater}
asd

\subsection{Relatert arbeid (1-2 sider)}
Her er hva andre folk har gjort.

\subsection{Konklusjoner og videre arbeid (0.5 side)}
Hva kan gjøres videre.

\subsection{Introduksjon, maskinlæring, måloppnåelse, design av eksperiment}
asd





