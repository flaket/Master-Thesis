%%=========================================
\section{Gestegjenkjennelse med fotodioder}
%%=========================================
\subsection{Introduksjon / Problemet (1-2 sider)}
{\color{red} Omskriv: FORKLAR SOM OM VED WHITEBOARDET!}

Problemet vi utforsker er å styre det smarte hjemmet på en naturlig og effektiv måte {\color{red} bruk et eksempel}. I et smart hjem skal det naturligvis være mulig å styre den gamle funksjonaliteten, som å skru av og på lys, men det vil også være andre styringsmuligheter; persienner og garasjedører skal opp og ned, temperatur skal stilles, dører skal låses opp og igjen og elektriske apparater skal kunne kontrolleres. Den gamle måten for å realisere dette er ved individuelle knapper og brytere for hver funksjonalitet.  {\color{red} BILDE AV VEGG MED KNAPPE-PANEL} Ettersom antallet enheter som kan styres stadig øker ender vi opp med en økende mengde brytere og det oppstår problemer rundt plass, estetikk og vanskeligheter med å lære hvilke brytere som hører til hvilket apparat. En løsning til dette er å samle styringskontrollen i en app, tilgjengelig på nettbrett og smarttelefoner. {\color{red} BILDE AV TYPISK STYRINGSAPP PÅ NETTBRETT} {\color{red} naturlig? effektivt?}. Bruken av gester for styring er utforsket i en rekke prosjekter. Tilnærmingen blant disse er RGB-kameraer, dybdekameraer og algoritmer for datasyn. Sammen kan disse prosessere bildene og danne grunnlaget for et system som blant annet kan gjenkjenne avanserte gester. Vel, bruken av kameraer har visse problemer i hjemmescenariet. Brukerundersøkelser {\color{red} refererfef} har vist at de færreste ønsker overvåking i smarte hjem. Selv dersom det kan garanteres at dataene fra kameraene holdes lokalt kan følelsen av at personvernet er utsatt være nok til at brukerene vil holde seg unna.

Dette kapittelet har følgende bidrag:
\begin{itemize}
\item Jeg viser at en gestesensor i form av enkle fotodioder er et tilstrekkelig medium for enkel brukerinteraksjon {\color{red} fwd ref}.
\item Jeg viser at maskinlæring kan benyttes for å lære et system å forstå enkle gester og at dette er et  alternativ til å programmere forståelse av ulike gester {\color{red} fwd ref}.
\item Jeg viser at enkle og kjente maskinlæringsalgoritmer kan benyttes for å forstå 10 ulike gester med 96\% suksess, og at det kreves svært få treningstilfeller. {\color{red} fwd ref}.
\end{itemize}

\subsection{Min idé (2 sider)}
{\color{red} IDEEN} Dette avsnittet presenterer hovedbidraget for dette kapitellet. 
Dersom man ønsker å tilby styringsmuligheter på en eller flere vegger i et hus kan enkle gestesensorer benyttes i steden for et stort panel av knapper og dimmere. {\color{red} BILDE AV SENSOR med størrelsesmål / sammenlignet med kjent, lite objekt.} Selve sensoren er på størrelse med et knappenålshode og vi kan dermed forestille oss at designere kan komme opp med produktimplementasjoner som enten forsvinner inn i hjemmiljøet eller synes tydelig, men er praktisk og estetisk veldesignet. Sensoren merker at en hånd eller et annet objekt befinner seg foran den ved å sende ut et svakt infrarødt signal som reflekteres og detekteres dersom signalet er sterkt nok når det returnerer. Dette vil bare skje dersom objektet er opptil 20 cm unna sensoren. Dette betyr at gester kun forstås dersom de utføres rett foran sensoren. I motsetning til gesteforståelse gjennom bruk av kameraer er dette altså en langt mindre påtrengende måte å "lytte" etter input fra brukerne. Brukere kan være sikre på at de hverken overvåkes eller at personvernet deres på noen måte brytes. En slik gestesensor fungerer rett og slett kun som en multifunksjonell knapp.

{\color{red} BILDE AV GEST}
Figur x viser en typisk gest der brukeren sveiper hånda foran sensoren som befinner seg på veggen. Vi kan forestille oss at denne sveipende bevegelsen foran denne sensoren betyr å skru av lyset i rommet sensoren befinner seg i. Men kanskje gesten betyr å skru på radioen, lukke gardinene eller starte kaffemaskinen. Det er ingen begrensning på hva en enkelt gest aktiverer i form av funksjonalitet.

Når en gest utføres må enten rådataene fra sensoren eller en forståelse av dataene sendes til en maskin som har ansvaret for å styre apparatene i hjemmet. Sensoren må være knyttet til en mikrokontroller eller tilsvarende som har ansvaret for å sende dataene videre. Dette kan enten være gjennom kabel eller trådløst. Det er mulig å programmere en mikrokontroller til å skille mellom sensordataene og forstå seks ulike gester {\color{red} ref}. Dette er i seg selv bra og betyr at en enkelt sensor kan fungere som en multifunksjonell knapp med minimum seks forskjellige kommandoer. Hvis man også utnytter at kombinasjonen av flere gester etter hverandre kan bety egne kommandoer er styringsmulighetene mange. Det finnes et alternativ til å programmere inn hva de ulike dataene skal tolkes som. Alternativet er å sende rådataene til en kraftigere maskin som kan benytte den spennende teknikken kjent som maskinlæring til å forstå enda flere ulike gester, med god sannsynlighet for suksess.

Maskinlæring handler om å la maskinen lære fra data. Dette kan enten være et forsøk på å finne ukjente sammenhenger i dataene den mates med eller det kan være å lære seg sammenhenger mellom dataeksempler og etiketter/klasser. Det er dette sistnevnte scenariet vi er interessert i. Vi kan mate maskinen med data fra en gest og samtidig gi informasjon om at dataene maskinen akkurat fikk betyr "sveip til høyre". Dermed kan maskinen danne en knytning mellom dataene som kom inn da vi sveipet til høyre og etiketten "sveip til høyre". Med tilstrekkelig treningseksempler fra de ulike gestene skal maskinen kunne lære seg forskjellene mellom de ulike gestene. Dermed vil den være i stand til å gjette riktig på hvilken gest vi utfører ved en senere anledning.

Prototypen jeg har utviklet er trent med 50 ulike dataeksempler på hver av de 10 ulike gestene. Systemet er i stand til å klassifisere nye gester korrekt 96\% av tiden. {\color{red} Hvor bra er dette?} 

{\color{red} BILDE AV DE 10 (14?) ULIKE GESTENE}


\subsection{Detaljer (5 sider)}
Her omtaler vi detaljer rundt sensoren, mikrokontrolleren, håndtering av data og bruk av maskinlæring.

\subsubsection{Hardware}

\subsubsection*{APDS9960}
APDS9960 er en sensor som tilbyr måling av lys og farge, oppdagelse av nærhet og gestegjenkjennelse.

\subsubsection*{Sparkfun}
Prototypen er utviklet med Sparkfun's innpakning av APDS9960-sensoren. Produktet fra Sparkfun gjør APDS9960-sensoren tilgjengelig for enkel prototyping ved å bryte ut ulike pinner.

\subsubsection*{Arduino}
I tillegg har de gjort tilgjengelig programvare til arduino-plattformen. De ulike programmene kan lastes opp til en arduino med gestesensoren tilkoblet og tilbyr ulik funksjonalitet. 

\subsubsection{Data-prosessering}
Hente data fra sensoren. Hva slags type data? Hva må gjøres av preprossesering? Valg av algoritmer.

Seriell data sendes fra mikrokontrolleren til datamaskinen som lytter på riktig port. Sensoren sender data så lenge ir-signalet reflekteres. Dette betyr at en gest som tar lengre tid skaper mer data. Et rolig sveip over sensoren med hele hånda kan skape 100-200 datapunkter. Et raskt flikk med to fingre skaper 16-32 datapunkter. En gest som involverer å holde hånda foran sensoren i et sekund eller mer skaper flere hundre datapunkter. For å benytte de planlagte klassifiseringsteknikkene må hvert av dataeksemplene inneholde like mange datapunkter. Hver input må ha like mange features.

Det finnes ulike metoder for å løse dette problemet. En av de er å bestemme et øvre antall maksimale features som skal tas med. Inputeksempler som ikke har tilstrekkelig med features blir paddet med 0-verdier. Å sette en maksgrense på features kan føre til at man mister viktig data fra gester som tar lengre tid å utføre. For et inputeksempel fra en rask gest vil mange features være 0. Dette kan påvirke effektiviteten til læringsalgoritmen. Et annet alternativ er å velge et fast antall features vært inputeksempel skal ha og så mappe inputeksempelet til denne featurevektoren. Dersom input har få datapunkter blir featurevektoren sparsom med data spredt jevnt utover og med 0-verdier i mellom. Dersom input har mye data vil hver feature i featurevektoren være et gjennomsnitt av en viss mengde datapunkter.

{\color{red} BILDE AV DATA SOM MAPPES TIL LITEN FEATUREVEKTOR, BÅDE LITE INPUT OG MYE INPUT.}

Jeg valgte å lage vektorer med 128 features. Dette tallet ble valgt basert på antall datapunkter som genereres ved ulike aktuelle gester. 128 features er nok til å gi tilstrekkelig detaljer selv ved gester som tar noe lengre tid og samtidig ikke så mange at raske gester skaper for sparsomme resultater. Featurevektoren normaliseres ved å mappe de mulige sensorverdiene 0-255 til 0-1.0. 
{\color{red} pythonkode for å lage featurevektor.}

\subsubsection{Overvåket maskinlæring: Klassifisering}
Arthur Samuel (1959). Maskinlæring er gir datamskiner evnen til å lære uten å bli eksplisitt programmert.

Tom Mitchell (1998). Et dataprogram sies å lære fra erfaring med hensyn på en oppgave og et ytelsesmål, dersom ytelsen på oppgaven, som målt av ytelsesmålet, øker med erfaring.

Den generelle modellen er å mate et treningssett av datapunkter til en læringsalgoritme. Algoritmen danner en hypotese om hva slags modell som best beskriver dataene. Denne hypotesen kan så benyttes til å gjøre gjetninger på nye datapunkter.
{\color{red} Traning set -> Learning Algorithm -> Hypothesis -> Prediction on test data}

Hypotesen avhenger av egenskapene i datapunktene og er en funksjon av parameterne i treningsdataene $x$:

\[ h_\theta(x) = \theta_0 + \theta_1x_1 + \theta_2x_2 + ... \theta_nx_n\]

En vellykket hypotese vekter $\theta$-verdiene godt for å maksimere ytelsen på algoritmen. Så hvordan velger man disse $\theta$-verdiene? Vi ønsker $\theta$-verdier slik at \(h_\theta(x)\) er nære $y$ for treningseksemplene \((x,y)\).

{\color{red} Basert på dataene er det to algortimer som bør fungere bra: logres og svm...}

\subsubsection{Logistisk regresjon}
asd

\subsubsection{Støttevektormaskiner}
asd

\subsubsection{Eksperimentsutførelse}
asdasdasdasdasd

\subsubsection{Resultater}
asdasdasdasdasd

\begin{table}[h!]
\centering
\begin{tabular}{|| c c ||}
\hline
\% Korrekt klassifisering & Algoritme \\ [0.5ex] 
 \hline\hline
 0.96 & SVM \\ 
 \hline
 0.958 & SVC \\
 \hline
 0.94048 & Logistisk regresjon \\ [1ex]
 \hline
\end{tabular}
\caption{Gjennomsnittlige resultater klassifisering}
\label{table:results}
\end{table}

{\color{red} andre algoritmer som kan være aktuelle med flere samples, graf som viser resultater for 100, 250 og 500 samples}

\subsection{Relatert arbeid (1-2 sider)}
Her er hva andre folk har gjort.

\subsection{Konklusjoner og videre arbeid (0.5 side)}
Hva kan gjøres videre: online læring, flere gester (er det nyttig?), større featurevektor (enn 128)




