%%=========================================
\section[Metoder \& Implementasjon]{Metoder \& Implementasjon}
%%=========================================
\subsection{Min subjektivitet}
\subsection{Fire hypoteser}
\begin{itemize}
\item Maskinlæring kan benyttes for å gi enkle sensorer en bedre forståelse av ulike gester enn det en programmerer kan.
\item Kombinasjonen av enkle gester og begrenset tale er både en effektiv og naturlig måte å kontrollere hjemmet på, og det er uproblematisk å implementere et slikt system.
\item Flere enkle gestesensorer i kombinasjon med hverandre, med talegjenkjenning og med nærhetssensorer kan danne et grunnlag for enda mer kapable interaksjonsmetoder.
\item For å best mulig tilby brukere et informasjonssystem for det smarte hjemmet bør systemet være drevet av kontekst, ikke interaksjon.
\end{itemize}

\subsection{Design av eksperimenter}
\subsection{Implementasjon}
maskinlæring, multimodalitet, kombinasjoner, kontekstdrevet ui
