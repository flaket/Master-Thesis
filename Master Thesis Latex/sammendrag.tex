%%=========================================
%\addcontentsline{toc}{section}{Sammendrag \& konklusjoner}
\section*{Sammendrag og konklusjoner}
Dette prosjektet har hatt som mål å utforske ulike brukergrensesnitt i smarte hjem. Spesielt har fokuset vært på å utforske \emph{intelligente brukergrensesnitt}; alternative interaksjonsmetoder som kan ha økt effektivitet og som oppleves som naturlige. Dette ble utfordret gjennom å spørre hva slags programvare man ønsker i smarte hjem. Basert på logisk argumentasjon og empiriske bevis fra to studier viser jeg at brukere av smarte hjem ønsker å lære om hjemmets tilstand og å utøve styring over hjemmet, mens privatliv og personvern ivaretas. Med dette utgangspunktet utforsket jeg hvordan maskinlærte gester og begrenset tale kan benyttes for å gi kommandoer, og hvordan informasjon om hjemmet best mulig kan presenteres til brukeren. Her ble det fokusert på å vise så mye informasjon på en optimal måte gjennom teknikker fra grafisk design. Regler for hvordan å vise den mest essensielle informasjonen ble utforsket og implementert i et dynamisk grensesnitt, som forandrer seg basert på dataene fra hjemmet.

Oppgaven har vist at maskinlæring kan benyttes for å gi enkle sensorer en mer detaljert forståelse av abstrakte gester, enn det som kan eksplisitt programmeres. Jeg har vist at begrenset tale og gester i kombinasjon er en aktuell måte å styre hjemmet på. Og jeg har vist hvordan et tradisjonelt brukergrensesnitt kan forbedres ved å la det drives av kontekstinformasjonen i hjemmet og ikke av brukerens interaksjoner med programvaren.