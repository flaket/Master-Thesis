%%=========================================
\section{Multimodal interaksjon gjennom tale og gester}
%%=========================================
\subsection{Introduksjon / Problemet (1-2 sider)}
{\color{red} FORKLAR SOM OM VED WHITEBOARDET!}

Igjen prat om naturlig interaksjon. Hva er mer naturlig enn tale? Drømmen om å kommunisere kontinuerlig med datamaskinen gjennom tale. Er dette ønskelig i et hjem-scenario? Eksempelproblemet med de nye Samsung-tv'ene. Kanskje bruksområdet til tale bør begrenses til enkle kommandoer: "skru på lyset", "skru av tv'en". Selv om kommandoene kanskje bør være enkle kan det bli mange av dem, som jeg påpekte i kapitell 2. Et system som skal forstå tale må dermed kunne gjenkjenne og skille mange talekommandoer og vi er dermed inne i nlp-land. De fleste kommersielle systemer benytter seg av et aktiveringsord; man må si "OK Google" eller lignende for å aktivere talegjenkjenningen. 

Dette kapittelet har følgende bidrag:
\begin{itemize}
\item Viser at kombinasjonen av gester og tale hjelper talegjenkjenning ved å forstå domenet {\color{red} fwd ref}.
\item Viser at  {\color{red} fwd ref}.
\item Implementert et grafisk grensesnitt for å simulere effekten av kommandoer i simulert hjemmiljø {\color{red} fwd ref}.
\end{itemize}

\subsection{Min idé (2 sider)}

\subsection{Detaljer (5 side)}
Tale, gester, multimodalitet

\subsection{Relatert arbeid (1-2 sider)}

\subsection{Konklusjoner og videre arbeid (0.5 side)}

